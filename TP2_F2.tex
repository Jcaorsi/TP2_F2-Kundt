% ******************************
%* FILE CONFIGURATION
% ******************************

\documentclass[12pt, a4paper]{article}

% Enables Spanish language support and table naming
\usepackage[spanish, es-tabla]{babel}

% Enables creation of hyperlinks in the document
\usepackage{hyperref}

% Enables inclusion of images and figures
\usepackage{graphicx}

% Provides multirow functionality for tables
\usepackage{multirow}

% Provides better control for floating elements like tables and figures
\usepackage{float}

% Sets page margins
\usepackage[left=2.5cm, right=2.5cm, top=2cm, bottom=2cm]{geometry}

% Allows customization of headers and footers
\usepackage{fancyhdr}
\pagestyle{fancy}
\fancyhf{} % Clears all header and footer fields
\fancyfoot[C]{\thepage} % Centers the page number at the footer
\renewcommand{\headrulewidth}{0pt} % Removes the header line
\renewcommand{\footrulewidth}{0pt} % Removes the footer line

% Forces the same style on first page
\fancypagestyle{plain}
{
  \fancyhf{}
  \fancyfoot[C]{\thepage}
  \renewcommand{\headrulewidth}{0pt}
  \renewcommand{\footrulewidth}{0pt}
}

% Adjusts table border width
\setlength{\arrayrulewidth}{0.5mm}

% Adjusts spacing between table columns
\setlength{\tabcolsep}{5pt}

% Changes the name of the table caption
\def\tablename{Tabla}


% ******************************
%* NEW COMMANDS
% ******************************

% Creates a custom numbered question command
\newcounter{order}
\newcommand{\order}[1]
{
  \par\vspace{2ex}
  \stepcounter{order}
  \noindent\textbf{\arabic{order}.} #1\par\vspace{1ex}
}


% ******************************
%* ARTICLE SECTIONS
% ******************************

\title{TP 2: Tubo de Kundt}
\author
{
  Caorsi Juan Ignacio, \href{jcaorsi@itba.edu.ar}{jcaorsi@itba.edu.ar} \\
  Dib Ian, \href{idib@itba.edu.ar}{idib@itba.edu.ar} \\
  Moschini Rita, \href{rmoschini@itba.edu.ar}{rmoschini@itba.edu.ar} \\
  Tamagnini Ana, \href{atamagnini@itba.edu.ar}{atamagnini@itba.edu.ar}
}

\date{Grupo 4 - 15/04/2025}

\begin{document}
\maketitle

%* FIRST QUESTION
\order{¿El micrófono mide variaciones de presión o desplazamientos del aire?}

(Juani)


%* SECOND QUESTION
\order{Determine la frecuencia del modo fundamental y la frecuencia de los siguientes tres armónicos.}

Se sabe que
\begin{equation}
    f_{n}=\frac{n\cdot v}{2L}
\end{equation}

donde $v$ es la velocidad del sonido, L el largo del tubo, $n$ el número de armónico y $f_{n}$ la frecuencia del armónico n. También sabemos que $v_{sonido}\simeq 330 \frac{m}{s}$ y que $L=0,5 m$
$\Rightarrow f_{1}=\frac{330 \frac{m}{s} }{2 \cdot 0,5m} \simeq 330 Hz $. Empleando este valor y la relación $f_{n}=n\cdot f_{1}$ se buscaron los valores teóricos de armónicos $f_{n}$. Luego, para encontrar los valores experimentales, se varió la frecuencia del generador de funciones; a medida que aumentaba la frecuencia del generador, se podía observar en el osciloscopio cómo aumentaba la amplitud de la señal hasta excederse la frecuencia $f_{n}$, entonces la amplitud comenzaba a decrementarse, para luego volver a incrementarse hasta llegar a $f_{n+1}$. Los armónicos eran las frecuencias para las cuales la amplitud rebotaba.
De esta manera, encontramos los siguientes datos:

\begin{table}[H]
    \centering
    \begin{tabular}{|c|c|c|}
    \hline
    \multirow{2}{2.1cm}{ Armónicos }
        & Valores Teóricos (Hz) & Valores Experimentales (Hz)\\
    \hline
    $f_1$  & 330  & 346 \\ \hline
    $f_2$  & 660 & 666 \\ \hline
    $f_3$  & 990 & 960 \\ \hline
    $f_4$  & 1320  & 1177 \\ \hline
    \end{tabular}
    \caption{Armónicos de la onda estudiada.}
    \label{tabla1}
\end{table}
Se puede observar que los valores experimentales de los armónicos difieren de los teóricos. Una conjetura sobre la razón de esta diferencia es que el tubo no es ideal (entre otras razones porque posee cierto grado de porosidad) entonces se disipa energía por fuera del sistema. Se puede notar esto porque el sonido escapa por fuera del tubo y llegó a oídos de los alumnos participantes en el experimento, lo cual en un sistema ideal no hubiera ocurrido.


%* THIRD QUESTION
\order{Calcule la velocidad del sonido}

\begin{table}[H]
    \centering
    \begin{tabular}{|c|c|}
    \hline
    \multirow{1}{2.1cm}{\centering $f_n$} 
        & $V_n$ ($\frac{m}{s}$) \\
    \hline
    $f_1$  & 346 \\ \hline
    $f_2$  & 333 \\ \hline
    $f_3$  & 320 \\ \hline
    $f_4$  & 294  \\ \hline
    \end{tabular}
    \caption{Velocidad del sonido obtenida en base a los armónicos hayados en la tabla \ref{tabla1}}
    \label{tabla2}
\end{table}

Luego buscamos el promedio de la velocidad del sonido, obteniendo aproximadamente $V_s = 323 \ \mathrm{m/s}$.


%* FOURTH QUESTION
\order{Determine el factor de calidad}

\begin{table}[H]
    \centering
    \begin{tabular}{|c|c|c|c|c|c|}
    \hline
    \multirow{1}{2.1cm}{\centering Armónicos} 
        & $V_\mathrm{max} (mV)$
        & $\frac{V_\mathrm{max}}{\sqrt{2}}$ $(mV)$ 
        & $f^+$ (s$^{-1}$) 
        & $f^-$ (s$^{-1}$)
        & $Q$ \\
    \hline
    $f_1$  & 160 & 113 & 357 & 337 & 17\\ \hline
    $f_2$  & 300 & 212 & 673 & 659 & 48\\ \hline
    $f_3$  & 212  & 150 & 980 & 945 & 27\\ \hline
    $f_4$  & 200  & 140 & 1249 & 1090 & 7\\ \hline
    \end{tabular}
    \caption{Datos obtenidos a partir de la experiencia y del reemplazo de los mismos en la ecuación del factor de calidad}
    \label{tabla3}
\end{table}

\end{document}
